\documentclass[12pt, a4paper]{article}
%\usepackage[swedish]{babel}
\usepackage[utf8]{inputenc}
\usepackage[T1]{fontenc}
\usepackage{hyperref}
\usepackage{url}
\usepackage{fullpage}
\usepackage{graphicx}
\usepackage[fleqn]{amsmath}
\usepackage{setspace}
\usepackage{graphicx}
\usepackage{siunitx}
\usepackage{parskip}
\usepackage{amsmath,amsfonts,amsthm, bm}
\usepackage{float}
\usepackage{wrapfig}
\usepackage[font={footnotesize,it}]{caption}
\usepackage{subcaption}
\usepackage{fancyref}
\usepackage{gensymb}
\usepackage{setspace}   
\usepackage{lipsum} 
\usepackage{gensymb}
\usepackage{xcolor}
\usepackage[export]{adjustbox}
\usepackage[utf8]{inputenc}
\usepackage{csquotes}
\usepackage[backend=bibtex,style=verbose-trad2]{biblatex}

%Från share latex
%\documentclass[12pt, a4paper]{article}
%\usepackage[swedish]{babel}
\usepackage[utf8]{inputenc}
\usepackage[T1]{fontenc}
\usepackage{hyperref}
\usepackage{url}
\usepackage{fullpage}
\usepackage{graphicx}
\usepackage[fleqn]{amsmath}
\usepackage{setspace}
\usepackage{graphicx}
\usepackage{siunitx}
\usepackage{parskip}
\usepackage{amsmath}
\usepackage{float}
\usepackage{wrapfig}
\usepackage[font={footnotesize,it}]{caption}
\usepackage{subcaption}
\usepackage{fancyref}
\usepackage{gensymb}
\usepackage{setspace}   
\usepackage{lipsum} 
\usepackage{minted}
\usepackage{gensymb}
\usepackage[export]{adjustbox}
\usepackage[utf8]{inputenc}
\usepackage{csquotes}
\documentclass{article}
\usepackage{enumitem,amssymb}
\newlist{todolist}{itemize}{2}
\setlist[todolist]{label=$\square$}
\usepackage{pifont}
\newcommand{\cmark}{\ding{51}}%
\newcommand{\xmark}{\ding{55}}%
\newcommand{\done}{\rlap{$\square$}{\raisebox{2pt}{\large\hspace{1pt}\cmark}}%
\hspace{-2.5pt}}
\newcommand{\wontfix}{\rlap{$\square$}{\large\hspace{1pt}\xmark}}


\DeclareMathSizes{12}{15}{11}{6}
%Den tredje är understreck_bokstav storlek
%Den andra är storleken


\begin{document}
\pagenumbering{gobble}
\begin{titlepage}
\centering
	\onehalfspacing
	\includegraphics[width=0.3\textwidth]{Logga_UU.png}\par\vspace{1cm}
	\vspace{0cm}
    \begin{spacing}{0.5}


	{\scshape\Large Signals and Transforms \par}
	\ignorespaces
	{\fontsize{50}{60}\bfseries Project\par}
	
    \end{spacing}

\vspace{2cm}
\begin{center}
Vladislav Bertilsson | Gustav Fridén | Felix Ljungkvist | Anton Nordström\\ 
\vspace{1 cm}
\end{center}
\vspace{2 cm}
    {\large\today \par}
    {\normalsize }
	\vfill

\end{titlepage}

\newpage
\section{Summarize} 

\newpage\tableofcontents

\pagenumbering{arabic}
\section{Summary}
\section{Introduction}
\section{Theory}

a.) The modulated signal $x_m [k]$ can be expressed as:
\newline
$$
x_{m}[k]=x_{b}[k] x_{c}[k]=x_{b}[k] A_{c} \sin \left(\Omega_{c} k\right)
$$

Where $\Omega_{c} = 2 \pi \cdot fb$ where the fb is the frequency band given to our group is 3475-3525 Hz. This leads to $\Omega_{c} = 7000 \cdot \pi$.
\newline
We want $x_{m}[k]$ to be continuous because its easier to perform the calculations on a continuous time signal.  We also know that:
$$
x_m(t) = 
\left\{\begin{array}{ll}
-A_{c} \sin \left(\omega_{c} k\right) & x_{b}(t)=-1 \\
A_{c} \sin \left(\omega_{c} k\right) & x_{b}(t)=1
\end{array}\right.
$$
Our signal of bit streams $x_b(t)$ can be described as a sum of rectangular pulses of width $T_b$. This leads to:
$$
x_b(t)=\operatorname{rect}\left(\frac{t}{T_{0}}\right)
$$
To get the complete signal $x_m(t)$ we multiply the carrier signal $x_c(t)$ with our bit stream signal $x_b(t)$.
We get the complete signal:
$$
x_m(t) = \operatorname{rect}\left(\frac{t}{T_{0}}\right) \cdot A_{c} \sin \left(\omega_{c} k\right)
$$
We can use fourier transform to get the spectrum of the signal $x_m(t)$. We set $A_c = 1$ for now:
$$
X_m(\omega) = T_0 sinc(\frac{\omega T_0}{2 \pi}) * \frac{1}{2 \pi} \pi [\delta(\omega + \omega _c)+\delta(\omega - \omega _c)] $$
$$
= \frac{1}{2} T_0 sinc(\frac{\omega T_0 }{2 \pi} - \omega _c) +  \frac{1}{2} T_0 sinc(\frac{\omega T_0 }{2 \pi} + \omega _c) 
$$
b) The frequency band given to our group was 4750–4850 Hz. This band has a peak at 4800 Hz. Therefore we choose a carrier frequency of 4800 Hz as our carrier frequency.

We set Tb equal to 0.01s because we have a frequency range of 100hz, and tb = 1/100. The bit rate is then 1/tb = 100 bits/ second

c.) Our group was given a maximum avarage power of 30 dBm. We have to choose a value of $A_c$ that fulfills this condition. We know that:
 \[
   P = 10^{\frac{P_{\text{dBm}}}{10}} \, \text{mW} = 10^{\frac{30}{10}} \, \text{mW} = 1000 \, \text{mW} = 1 \, \text{W}
   \]
Given the carrier signal \( x(t) = A_c \sin(\omega_0 t + \varphi) \), the average power \( P \) is given by:
   \[
   P = \frac{A_c^2}{2}
   \]
   Setting \( P = 1 \, \text{W} \):
   \[
   \frac{A_c^2}{2} = 1
   \]
   Solving for \( A_c \):
   \[
   A_c = \sqrt{2} \approx 1.414
   \]
d.)
Given the expressions:
\begin{align*}
y_{I,d}(t) &= y_m(t) \cos(\omega_c t) = A_r y_b(t - t_0) \sin(\omega_c t + \phi_r) \cos(\omega_c t), \\
y_{Q,d}(t) &= -y_m(t) \sin(\omega_c t) = -A_r y_b(t - t_0) \sin(\omega_c t + \phi_r) \sin(\omega_c t).
\end{align*}
We aim to simplify these expressions by eliminating the products between sine and cosine terms.
\subsection*{Simplifying \( y_{I,d}(t) \)}
First, expand \(\sin(\omega_c t + \phi_r)\) using the identity:
\[
\sin(\omega_c t + \phi_r) = \sin(\omega_c t) \cos(\phi_r) + \cos(\omega_c t) \sin(\phi_r).
\]
Substitute back into \( y_{I,d}(t) \):
\begin{align*}
y_{I,d}(t) &= A_r y_b(t - t_0) \left[ \sin(\omega_c t) \cos(\phi_r) + \cos(\omega_c t) \sin(\phi_r) \right] \cos(\omega_c t) \\
&= A_r y_b(t - t_0) \left[ \cos(\phi_r) \sin(\omega_c t) \cos(\omega_c t) + \sin(\phi_r) \cos^2(\omega_c t) \right].
\end{align*}
Next, use the trigonometric identities:
\[
\sin(\omega_c t) \cos(\omega_c t) = \tfrac{1}{2} \sin(2\omega_c t), \quad \cos^2(\omega_c t) = \tfrac{1}{2} [1 + \cos(2\omega_c t)].
\]
Applying these identities:
\begin{align*}
y_{I,d}(t) &= A_r y_b(t - t_0) \left[ \cos(\phi_r) \left( \tfrac{1}{2} \sin(2\omega_c t) \right) + \sin(\phi_r) \left( \tfrac{1}{2} [1 + \cos(2\omega_c t)] \right) \right] \\
&= \tfrac{A_r}{2} y_b(t - t_0) \left[ \cos(\phi_r) \sin(2\omega_c t) + \sin(\phi_r) + \sin(\phi_r) \cos(2\omega_c t) \right].
\end{align*}
Combine terms using the identity:
\[
\cos(\phi_r) \sin(2\omega_c t) + \sin(\phi_r) \cos(2\omega_c t) = \sin(2\omega_c t + \phi_r).
\]
Therefore:
\[
y_{I,d}(t) = \tfrac{A_r}{2} y_b(t - t_0) \left[ \sin(\phi_r) + \sin(2\omega_c t + \phi_r) \right].
\]
\subsection*{Simplifying \( y_{Q,d}(t) \)}
Use the identity for the product of sines:
\[
\sin(A) \sin(B) = \tfrac{1}{2} [\cos(A - B) - \cos(A + B)].
\]
Applying this to \( y_{Q,d}(t) \):
\begin{align*}
y_{Q,d}(t) &= -A_r y_b(t - t_0) \sin(\omega_c t + \phi_r) \sin(\omega_c t) \\
&= -A_r y_b(t - t_0) \left[ \tfrac{1}{2} \cos(\omega_c t - (\omega_c t + \phi_r)) - \tfrac{1}{2} \cos(\omega_c t + (\omega_c t + \phi_r)) \right] \\
&= -\tfrac{A_r}{2} y_b(t - t_0) \left[ \cos(-\phi_r) - \cos(2\omega_c t + \phi_r) \right].
\end{align*}
Since \(\cos(-\phi_r) = \cos(\phi_r)\), this simplifies to:
\[
y_{Q,d}(t) = -\tfrac{A_r}{2} y_b(t - t_0) \left[ \cos(\phi_r) - \cos(2\omega_c t + \phi_r) \right].
\]
\subsection*{Spectra of \( Y_{I,d}(\omega) \) and \( Y_{Q,d}(\omega) \)}
The expressions show that \( y_{I,d}(t) \) and \( y_{Q,d}(t) \) contain two components:
\begin{itemize}
    \item \textbf{Baseband Component}:
    \begin{align*}
    y_{I,d}^{\text{BB}}(t) &= \tfrac{A_r}{2} y_b(t - t_0) \sin(\phi_r), \\
    y_{Q,d}^{\text{BB}}(t) &= -\tfrac{A_r}{2} y_b(t - t_0) \cos(\phi_r).
    \end{align*}
    \item \textbf{Component at \( 2\omega_c \)}:
    \begin{align*}
    y_{I,d}^{2\omega_c}(t) &= \tfrac{A_r}{2} y_b(t - t_0) \sin(2\omega_c t + \phi_r), \\
    y_{Q,d}^{2\omega_c}(t) &= \tfrac{A_r}{2} y_b(t - t_0) \cos(2\omega_c t + \phi_r).
    \end{align*}
\end{itemize}

e.)
While it might seem sufficient to recover the baseband signal $y_b(t - t_0)$ by low-pass filtering either $y_{I,d}(t)$ or $y_{Q,d}(t)$ to eliminate the $2\omega_c$ components, this is ineffective because the baseband amplitudes in these signals depend on the unknown phase $\phi_r$. Specifically, the baseband component in $y_{I,d}(t)$ is scaled by $\sin(\phi_r)$, and in $y_{Q,d}(t)$ by $-\cos(\phi_r)$. Without precise knowledge of $\phi_r$, which may vary over time, relying on a single signal could result in significant signal attenuation or complete loss (e.g., when $\sin(\phi_r) = 0$ or $\cos(\phi_r) = 0$), making it unreliable to extract $y_b(t - t_0)$ from just one of the two signals.


\section{Results and Discussion}
\section{Conclussions}
\begin{thebibliography}{9}

\bibitem{latexcompanion} 
Michel Goossens, Frank Mittelbach, and Alexander Samarin. 
\textit{The \LaTeX\ Companion}. 
Addison-Wesley, Reading, Massachusetts, 1993.

\bibitem{einstein} 
Albert Einstein. 
\textit{Zur Elektrodynamik bewegter K{\"o}rper}. (German) 
[\textit{On the electrodynamics of moving bodies}]. 
Annalen der Physik, 322(10):891–921, 1905.

\bibitem{knuthwebsite} 
Knuth: Computers and Typesetting,
\\\texttt{http://www-cs-faculty.stanford.edu/\~{}uno/abcde.html}



\bibitem{Grout} 
Ian Grout. 
\textit{Digital Systems Design with FPGAs and CPLDs}, 2008.
\\\texttt{https://www.sciencedirect.com/topics/engineering/iir-filters}

\end{thebibliography}

\end{document}
